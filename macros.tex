\newcommand\nc\newcommand
\nc\rnc\renewcommand

%% https://www.conference-publishing.com/Help.php
\usepackage[utf8]{inputenc} 
\usepackage[T1]{fontenc}
\usepackage{microtype}

\usepackage{epsfig}
\usepackage{latexsym}
\usepackage{amsmath}
\usepackage{amssymb}
\usepackage{color}

\usepackage{subcaption}

\usepackage[us,12hr]{datetime}
\usepackage{setspace}

% bottom prevents floats (figures) below footnotes, while hang & flushmargin
% avoids indentation.
\usepackage[bottom,hang,flushmargin]{footmisc}

\nc\out[1]{}

%% \nc\noteOut[2]{\note{#1}\out{#2}}

%% To redefine for a non-draft
\nc\indraft[1]{#1}

\nc\note[1]{\indraft{\textcolor{red}{#1}}}

\nc\notefoot[1]{\note{\footnote{\note{#1}}}}

\nc\todo[1]{\note{To do: #1}}

\nc\eqnlabel[1]{\label{equation:#1}}
\nc\eqnref[1]{Equation~\ref{equation:#1}}
\nc\eqnreftwo[2]{Equations~\ref{equation:#1} and \ref{equation:#2}}

\nc\figlabel[1]{\label{fig:#1}}
\nc\figref[1]{Figure~\ref{fig:#1}}
\nc\figreftwo[2]{Figures~\ref{fig:#1} and \ref{fig:#2}}

\nc\seclabel[1]{\label{sec:#1}}
\nc\secref[1]{Section~\ref{sec:#1}}
\nc\secreftwo[2]{Sections~\ref{sec:#1} and~\ref{sec:#2}}
\nc\secrefs[2]{Sections \ref{sec:#1} through \ref{sec:#2}}

\nc\appref[1]{Appendix~\ref{sec:#1}}

%% The name \secdef is already taken
\nc\sectiondef[1]{\section{#1}\seclabel{#1}}
\nc\subsectiondef[1]{\subsection{#1}\seclabel{#1}}
\nc\subsubsectiondef[1]{\subsubsection{#1}\seclabel{#1}}

\nc\needcite{\note{[ref]}}

% \nc\myurl\texttt

%% For acmlarge, I have to put my figures side by side

%% http://tex.stackexchange.com/questions/5769/two-figures-side-by-side

\setlength{\fboxsep}{1ex}
\setlength{\fboxrule}{0.05ex} % gray or dotted might work better

%% \figone{fraction}{label}{caption}{content}
\nc\figoneW[4]{
\fbox{%
\begin{minipage}{#1\linewidth}
  \centering
  \setlength\mathindent{0ex}
  #4
  \vspace*{-5ex}
  \captionof{figure}{#3}
  \label{fig:#2}
\end{minipage}
}
}
\nc\figone{\figoneW{\stdWidth}}

%% %% \figo{fraction}{content}
%% \nc\figoW[2]{
%% \fbox{
%% \begin{minipage}{#1\linewidth}
%%   \centering
%%   \setlength\mathindent{0ex}
%%   #2
%% \end{minipage}
%% }
%% }

\nc\figo[1]{
\begin{figure}
\centering
#1
\end{figure}
}

\nc\figp[2]{\begin{figure}\centering #1 \hspace{-2ex} #2\end{figure}}

% Arguments: env, label, caption, body
\nc\figdefG[4]{\begin{#1}[tbp]
\begin{center}
#4
\end{center}
\caption{#3}
\figlabel{#2}
\end{#1}}

% Arguments: label, caption, body
\nc\figdef{\figdefG{figure}}
\nc\figdefwide{\figdefG{figure*}}

% Arguments: label, caption, body
\nc\figrefdef[3]{\figref{#1}\figdef{#1}{#2}{#3}}

\nc\figrefdefwide[3]{\figref{#1}\figdefwide{#1}{#2}{#3}}

%% %% \circdefW{frac}{label/file}{caption}{work}{depth}
%% \nc\circdefW[5]{\figoneW{#1}{#2}{#3 \stats{#4}{#5}}{\incpic{#2}}}
%% \nc\circdef{\circdefW{\stdWidth}}

\nc\stdWidth{0.46}

%% \figpair{frac1}{frac2}{label1}{caption1}{content1}{label2}{caption2}{content2}
\nc\figpairW[8]{
\begin{figure}
\centering
\figoneW{#1}{#3}{#4}{#5}
%\hspace{.05\linewidth}
%\hfill
\figoneW{#2}{#6}{#7}{#8}
\end{figure}
}
%% \figpair{label1}{caption1}{content1}{label2}{caption2}{content2}
\nc\figpair{\figpairW{\stdWidth}{\stdWidth}}

\nc\incpic[1]{\includegraphics[width=\linewidth]{figures/#1}}

%% \incpicW{frac}{file}
\nc\incpicW[2]{\includegraphics[width=#1\linewidth]{figures/#2}}

%%%%%%

%% \nc\symTwo[1]{\mathbin{#1\!\!\!#1}}
%% \nc\symThree[1]{\mathbin{#1\!\!\!#1\!\!\!#1}}

\setlength\mathindent{4ex}
\nc\db[1]{\llbracket#1\rrbracket}

\nc\smalltriangleup{\triangle}
\nc\smalltriangledown{\triangledown}

%% Double quote symbol
\nc\dq{\text{\tt\char34}}
%% Quoted haskell string with formatted content
\nc\hquoted[1]{\dq\!#1\!\dq}

\nc\sectionl[1]{\section{#1}\seclabel{#1}}
\nc\subsectionl[1]{\subsection{#1}\seclabel{#1}}

\nc\workingHere{
\vspace{1ex}
\begin{center}
\setlength{\fboxsep}{3ex}
\setlength{\fboxrule}{4pt}
\huge\textcolor{red}{\framebox{Working here}}
\end{center}
\vspace{1ex}
}

\usepackage{amsthm}
%% %% Roman font, and drop vertical spacing before & after.
\theoremstyle{definition} % remark

\newtheoremstyle{plainstyle}
  {\topsep} % Space above
  {\topsep} % Space below
  {} % Body font
  {} % Indent amount
  {\bfseries} % Theorem head font
  {.} % Punctuation after theorem head
  {.5em} % Space after theorem head
  {} % Theorem head spec (can be left empty, meaning `normal')

%% \theoremstyle{plainstyle}

\newtheorem{definition}{Definition}%[section]
\nc\deflabel[1]{\label{definition:#1}}
\nc\defref[1]{Definition \ref{definition:#1}}
\nc\defreftwo[2]{Definitions \ref{definition:#1} and \ref{definition:#2}}
\nc\defrefs[2]{Definitions \ref{definition:#1} through \ref{definition:#2}}

\newtheorem{theorem}{Theorem}%[section]
\nc\thmlabel[1]{\label{theorem:#1}}
\nc\thmref[1]{Theorem \ref{theorem:#1}}
\nc\thmreftwo[2]{Theorems \ref{theorem:#1} and \ref{theorem:#2}}
\nc\thmrefs[2]{Theorems \ref{theorem:#1} through \ref{theorem:#2}}

\newtheorem{corollary}{Corollary}[theorem]
\nc\corlabel[1]{\label{corollary:#1}}
\nc\corref[1]{Corollary \ref{corollary:#1}}
\nc\correftwo[2]{Corollaries \ref{corollary:#1} and \ref{corollary:#2}}
\nc\correfs[2]{Corollaries \ref{corollary:#1} through \ref{corollary:#2}}

\newtheorem{lemma}[theorem]{Lemma}
\nc\lemlabel[1]{\label{lemma:#1}}
\nc\lemref[1]{Lemma \ref{lemma:#1}}
\nc\lemreftwo[2]{Lemmas \ref{lemma:#1} and \ref{lemma:#2}}
\nc\lemrefthree[3]{Lemmas \ref{lemma:#1}, \ref{lemma:#2}, and \ref{lemma:#3}}
\nc\lemrefs[2]{Lemmas \ref{lemma:#1} through \ref{lemma:#2}}

\newtheorem{exercise}[theorem]{Exercise}
\nc\exclabel[1]{\label{exercise:#1}}
\nc\excref[1]{Exercise \ref{exercise:#1}}
\nc\excreftwo[2]{Exercises \ref{exercise:#1} and \ref{exercise:#2}}
\nc\excrefs[2]{Exercises \ref{exercise:#1} through \ref{exercise:#2}}

\definecolor{codesep}{gray}{0.85}
\nc\codesep[1]{
\begin{minipage}[b]{0ex}
\color{codesep}{\rule[1ex]{0.8pt}{#1}}
\end{minipage}}


%% For multiple footnotes at a point. Adapted to recognize \notefoot as well
%% as \footnote. See https://tex.stackexchange.com/a/71347,
\let\oldFootnote\footnote
\nc\nextToken\relax
\rnc\footnote[1]{%
    \oldFootnote{#1}\futurelet\nextToken\isFootnote}
\nc\footcomma[1]{\ifx#1\nextToken\textsuperscript{,}\fi}
\nc\isFootnote{%
    \footcomma\footnote
    \footcomma\notefoot
}

%% https://tex.stackexchange.com/questions/336854/typesetting-a-wider-bar-sign
\DeclareFontFamily{U}{mathx}{\hyphenchar\font45}
\DeclareFontShape{U}{mathx}{m}{n}{
      <5> <6> <7> <8> <9> <10>
      <10.95> <12> <14.4> <17.28> <20.74> <24.88>
      mathx10
      }{}
\DeclareSymbolFont{mathx}{U}{mathx}{m}{n}
\DeclareFontSubstitution{U}{mathx}{m}{n}
\DeclareMathAccent{\widebar}{0}{mathx}{"73} % "

\usepackage{footnotebackref}
\usepackage{hyperref}
